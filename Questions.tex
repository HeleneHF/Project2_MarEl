\subsection*{Questions Lecturer/TA}

\begin{itemize}
    \item Does looses have to accounted for?
    \item OK to 4 MW service load for two busses, to a total of 8MW service load in normal operation? 
    \item How much do they care about the load on the generators/thrusters?
    \item Minimum reliability would suggest we only need either two compartments. Justify 4 because that's the way it is. Double check that we can run with three failed compartments?? Or does two thrusters in this case mean two thrusters in separate compartments?
    \checkmark The usual is that 4 thrusters are operating on normal DP, right? One of the thrusters in each of the compartments. 
    \item Both three and four thrusters can be powered using only three generators (in our current concept). Is that okay for normal operation or is the normal to run 4, one in each compartment, for redundancy in each compartment separate form the others?
    \item How detailed should the thruster lay out be? Only as in the power point with two thrusters in each corner, write that they are azimuth and that you might want to make sure they are not interfering with each other?
    \item When explaining terms (point 3). In to what extent should we re-write the definitions? Some of them (failure for example) do not give much place for re-writing. Others can be enriched (for example redundancy). Is it okay to take everything from the table or do we have to add more?
    \item A standard heavy duty thrusters only have a maximum capacity of 6.5 MW! How are we supposed to have thrusters using a load of 17MW each? The same for Voith and Värtsilä.  
    \item How to take the "single failure", because it would lead to a condition of only 1 compartment on service and 3 working. Then we can lower the capacity of thrusters and gensets. But we loose redundancy. then: should we follow the rules strictly or go for the 4 independent compartments configuration?
    \item the required propulsion power of 42 MW it too high, that would lead to a 4 groups of 3 thrusters of 6 MW each (the most powerful in the market is 6,5 MW).
    \item During operation after "worst-case failure" if we operate with only 2 gensets and 2 thrusters we need to double the generation and propulsion capacity as if we worked with 2 compartments fully operational in service. Then: should we double the capacity of generation and propulsion for having more redundancy? (redundancy on each compartment). 
    \item Should we draw new drawings? or can we make use of the ones given in the lecture slides and compendiums from other courses?
    \item Voltage levels on busbars: in accordance to the lecture notes: 11 kV. Is that a recommendation or is on the Rules? we could not find it.
\end{itemize}


Questions to the teachers - Round 2

\begin{itemize}
    \item NORSOK or IEC for citing MSB's voltage level 
    \item Thrusters' MSBs voltage (11kV regardless power?)
    \item recommended load level for service transformers
    \item Is it OK for the gensets to work between 10 and 90\% load? or should we specify other range?
    \item power factor for the thrusters' transformers
    \item for 8d) should we use the 3-phase power formula? (P=3 V I cos(phi)) is it OK to to work with 0,8 PF or should we assume other?.
    \item for 8f) remaining power: only active or should we calculate the complex?
\end{itemize}