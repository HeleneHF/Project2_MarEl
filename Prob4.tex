\section*{Problem 4}

As per DNV GL Rules for Classification of Ships \cite{RulesShipsDNVGLPart6Chap3} a \textit{power management system (PMS)} is required for both DYNPOS(AUTR) and DYNPOS(AUTRO) (DP Class 2 and DP Class 3).

\begin{comment}
Requirements (Section 1.8.4 - Power Management):

\begin{itemize}
    \item Automatic PMS, operating with both open and close bus-tie breakers.
    \item Shall be redundant   
    
    8.4.3 A failure in the power management system shall not cause alteration to the power generation, and
shall initiate an alarm in the DP control centre.

8.4.4 It shall be possible to operate the switchboards in manual as required for the main class, with the
power management system disconnected.

8.4.5 Means shall be implemented in order to prevent overloading of the power plant, e.g. by use of
interlocks, thrust limitations or other means. Means shall also be implemented to prevent reactive overload.
In case trust is reduced by any other system than the DP-control system this shall be communicated to the
DP-control system.

8.4.6 Overload, caused by the stopping of one or more generators subject to common mode failure, shall not
create a black-out. Reduction in load, e.g. thruster pitch or speed reduction shall be introduced to prevent
blackout and enable standby generators to come online.

8.4.7 When generators are running in parallel there shall be protection systems able to detect failures
that may result in a full or partial black-out situation and effectuate actions to prevent such incidents. The
effectuated actions shall be so that the consequence of the failure is minimized. This means that for failures
where the system has sufficient time the faulty component shall be tripped before a full or partial black-out
situation occurs. Such protection systems shall be independent from automatic voltage regulators (AVRs) and
engine governors.

8.4.8 When the system is operating with closed bus-tie breakers between switchboards belonging to
different redundancy groups and the first action performed by the protection system does not remove the
failure, or it is not performed (e.g. due to hidden failure), the protection system shall be able to execute
alternative actions to isolate the faulty component or system before the failure effect could propagate from
one system to another.
\end{itemize}
\end{comment}

Functions (Section 1.8.4.2 - Automatic Functions of the PMS):

The PMS shall be capable of performing the following automatic functions:

\todo[inline]{COPY-PASTE SO FAR}

\begin{itemize}
    \item load dependent starting of additional generators.
    \item block starting of large consumers when there is not adequate running generator capacity, and to start up generators as required, and hence to permit requested consumer start to proceed
    \item it shall be possible to set a minimum number of connected generator sets in each redundancy group
    \item black-out recovery on individual switch-board sections by starting of generators and recovery of full automatic thruster control from DP within 45 seconds after black-out. This means that the DP-control systems shall receive ready signal within 45 seconds.
\end{itemize}