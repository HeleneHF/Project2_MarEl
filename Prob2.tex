\section{Problem 2}
\subsection{Problem 2a)}

In order to fulfill the Equipment Class 3 required by the Petroleum Safety Authority Norway (as specified in Section \ref{problem_1a}) and in accordance to the DNV GL regulations \ref{Sec:Equivalent_Notations} \cite{RecommendedPractices_DP_DNVGL}, the notation will be DYNPOS(AUTRO). The tables in appendix \ref{Sec:Equipment_Requirements} and \ref{Sec:Class_Notations} specifies the DNV GL requirements for each part of the system that is needed to be classed as DYNPOS(AUTRO). 

To summarize, there should be redundancy in all active components and there should be physical separation of compartments. The system must be designed such that the vessel is able to maintain it's position long enough to safely terminate the work in progress after a single failure. This would correspond to for example ending a drilling operation in our case. There should be enough remaining thrust and power to maintain position after a worst-case failure. The system must also be designed such that minimum operator interference is required to switch on redundant equipment and this equipment should be available immediately \cite{RulesShipsDNVGLPart6Chap3}.
%\todo{This is all from the same source and the same as in the table. Everything is just rephrased like in question 3. Not sure if this is ok.}


%Such notation sets as a design requirements:

% \begin{itemize}
%     \item Redundancy of all active components (classification societies interpret this differently).
%     \item Physical separation of the components.
%     \item Time to terminate: Be able to maintain station long enough to safely terminate the work in progress after a \textit{single failure}.
%     \item The equipment intended to provide redundancy is available immediately and with a minimum of operator intervention.
%     \item Adequate remaining power and thrust after a \textit{worst-case failure}
% \end{itemize}

%%%% OLD %%%
% In order to fulfill the Equipment Class 3 required by the Petroleum Safety Authority Norway (as specified in \nameref{problem_1a}) and in accordance to the DNV GL regulations \ref{Sec:Equivalent_Notations} \cite{RecommendedPractices_DP_DNVGL}, the notation will be DYNPOS(AUTRO). The tables in appendix \ref{Sec:Equipment_Requirements} and \ref{Sec:Class_Notations} specifies the DNV GL requirements for each part of the system that is needed to be classed as DYNPOS(AUTRO). 

% Such notation sets as a design requirements:

% \begin{itemize}
%     \item Redundancy of all active components (classification societies interpret this differently).
%     \item Physical separation of the components.
%     \item Time to terminate: Be able to maintain station long enough to safely terminate the work in progress after a \textit{single failure}.
%     \item The equipment intended to provide redundancy is available immediately and with a minimum of operator intervention.
%     \item Adequate remaining power and thrust after a \textit{worst-case failure}
% \end{itemize}

% \todo[inline]{Keep reading recommended practices from page 22}
